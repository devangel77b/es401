\documentclass[10pt,courier]{navymemo}

\newcommand{\myroot}{../..}
\usepackage{\myroot/course}

\author{Dennis Evangelista}
\title{ES401 Advisor Policy and Tasking Letter}
\navysubj{Tools for biological study enabled by systems engineering; advisor policy and request for action}
\navyfiling{1531}
%\navyserial{16-0001}
\date{3 SEP 2018}
%\navymarking{UNCLASSIFIED}

\usepackage{designature}

\begin{document}
\makedateblock{}

\MEMORANDUM{}

\begin{navyletterheader}
\navyfrom{Assistant Professor Evangelista}
\navyto{ES401 Advisees, Biomechanics Capstone Team 2019}
\navyskip{}%

\navysubjline{}%
\navyskip{}%
\navyref{refa}{ACDEANINST 1531.58 (Administration of Academic Programs)}
\navyref{refb}{WEAPS\&SYSENGRINST 5400.7H (Teaching Methods \& Practices)}
\navyref{refc}{USNAINST 1531.53 (Policies Concerning Graded Academic Work)}
%\navyskip{}%
%\navyencl{encl1}{Syllabus, \usnaCourseNumber\ (\usnaCourseName), \courseTerm}
\end{navyletterheader}

\section{} This letter directs the Biomechanics Capstone Team 2019 (MIDN 1/c Edwards, Hall, Martin, Nemani) to commence concept studies for tools for biological study enabled by autonomy, machine vision, robotics and control, or drones. The work will be developed as part of ES401 (Engineering Design Methods; instructor CAPT Hurni) and will continue in the spring as ES404 (Systems Engineering Capstone).  Per references (\ref{refa}) and (\ref{refb}), this memorandum sets forth the advisor's policy for the \courseTerm\ session of \usnaCourseNumber, supplementing the course policy provided by your \usnaCourseNumber\ instructor.  The information here supplements the basic guidance provided in references~(\ref{refa}) through (\ref{refc}).  Successful completion of the sequence ES401-ES404 is \textbf{required in order to graduate.}

\section{Discussion} Autonomy, machine vision, robotics and control, and drones are all of growing importance in naval warfare and in civilian fields. They are just beginning to be introduced into basic scientific research in biology and ecology, but have great potential, especially in tasks that mirror military missions (count or identify a species of interest; surveill an area and note changes with time; provide persistent imagery; return samples; reach difficult-to-access areas; etc.) As such tasks are often linked with field work, they provide a robust challenge to designers, also mimicking many of the concerns faced in designing for military applications.

This project, formally, is open to any use of autonomy, machine vision, or other systems engineering core competency to accomplish some significant biological or ecological study task of interest. Use of a drone or mobile robot is not a requirement (though the team has indicated great interest in using drones). For example, a satisfactory project could develop a machine vision-enabled observatory for remote Antarctic or undersea use; could examine use of networked trail cameras in censusing rare populations; or could develop wearable backpacks/sensors and Big Data processing for animal motion, acoustic,or behavior studies.

\section{Application of drones} In preliminary discussions, the team has indicated great interest in using drones to perform a science mission. There are several projects and scientists identified as customers for application of drones to ecological studies. For example, Dr Bezy (UNC Biology) surveying turtle populations and nesting sites in Costa Rica; Dr Guilliams (Santa Barbara Botanical Garden) surveying botany in vernal pool areas; botanical sample return from inaccessible places; Asst Profs Wargula and Johnson (NAOE) using imagery to study coastal oceanography, transport, and ecology. These scientists have already been in contact and have already provided examples of customer stated requirements for their missions.
\subsection{} If the team is gravitating towards a drone project, they are directed to carry a potentially lower risk non-drone option during the early stages of ES401 for comparison purposes.
\subsection{} The team should use customer inputs as a guide, but may also survey other scientists to gauge how much functional requirements vary among science projects and allow use of their designs to accomplish additional future missions. 
\subsection{} The team is reminded of the need to appropriately scope their project; with four team members, approximately two major new developments can be expected/attempted. The team is at liberty to decide where major efforts should be spent, and may make use of COTS components to keep the scope of new development reasonable:  airframe; control system; sample return; machine vision; mission planning; logistics for the field. 

\section{Capstone Objectives and Learning Outcomes}  As the capstone experience for the Systems Engineering major, your capstone will draw upon the entirety of your engineering education to date.  At completion, you will:
\subsection{} Demonstrate new skills attained in support of the capstone project;
\subsection{} Scope and plan an engineering project of intermediate size, based on engineering insight gained during the execution of the capstone;
\subsection{} Demonstrate successful application of skills taught throughout the program, including troubleshooting, testing, evaluation, team building, decision making, and conflict resolution;
\subsection{} Apply the fundamentals of cost, schedule, and project management learned in ES401 to a novel project;
\subsection{} Demonstrate and analyze and engineering project at its completion, including evluation of both the success of the project and the quality of the initial plan and design;
\subsection{} Demonstrate skills associated with execution of a project test plan, and articulate the need for, and critical elements of, a good test plan;
\subsection{} Construct a complete and professional technical report and presentation that capture the process and results of the capstone effort, including sufficient data and detail to enable review and follow-on work (potentially by others).

\section{Additional Advisor Discussion} In order to successfully complete the Capstone goals above, the team will need to quickly work through concept and preliminary designs and develop mockups and demonstrations. 
\subsection{} The team will work to focus requirements, objectives, and functional requirements on a mission or set of missions of their choosing. The team may opt to split missions among different modules or different sub-designs to reduce overall risk. 
\subsection{} While ES401 formally typically directs only benchtop demos near the end of the class, chances of success will be maximized if the team begins developing mockups, demonstrations, and limited alpha prototypes as soon as possible (see also, below). 

\section{Resources} 
\subsection{} Lockheed Martin, via the Naval Academy Foundation, donated funds to support the Biomechanics Capstone in 2018, of which \$5876 has been rolled into FY19, to be split between your project, plus Toth's and Descour's honors projects. Additional gift funds have been identified for biomechanics (Evangelista and Jaramillo) in FY19, which may be made available to your project as needed. 
\subsection{} We are now offering ES281C (Intro to Drone Technology) as a School of Drones class to provide basic UAS training; Mr.~Martin is in the class. In addition, ENS Garcia has been directed to support getting your team into the air with quadrotors (Gremlin; 250; or the two Systems Ball Flamewheels) or fixed wing UAS (FT Dart; Mini Arrow; or Spear) as soon as possible, in advance of purchasing additional materials. 
\subsection{} A Pixhawk Mini, Raspberry Pi, and Picam cameras are available for testing in advance of purchasing final hardware. 
\subsection{} A Lenovo laptop configured with Linux Ubuntu 18.04.1 LTS, OpenCV 3, Mission Planner, and QGroundControl will also be made available.
\subsection{} Teammembers intending to work on payloads should obtain Solidworks through USNA, for design of 3D printed parts; as well as ExpressPCB or Eagle CAD for circuit boards. 
\subsection{} Mr.~Nemani is registered for ES495 and will prepare a poster targeted at the Society for Integrative and Comparative Biology conference; ES495 provides funds for midshipmen who are presenting to attend conferences.
\subsection{} Due to similarities in requirements, you may wish to perform benchmarking and cross-project review with the USNA Small Unmanned Aerial System (SUAS) team as well as the Squads with Autonomous Teammates (SWAT-C) Capstone team (Kubena, Mauricio, Forand and Bond). 

\section{Project management requirements from the advisor}

\subsection{} For major design decisions, I expect clear and concise documentation in the form of a Navy memo following the fact-discusison-action pattern. Additional calculations, drawings, or design information may also be needed. Major design decisions may need to include concurrence of other stakeholders as needed. Depending on the path taken by the project, intermediate design products (such as weight reports; load analyses; estimates of actuator force or torque; budgets; Gantt charts; etc) may require the same level of documentation. 

\subsection{} Major deliverables (e.g. concept design review presentation, preliminary design review presentation, report, SICB poster) will be routed to me for comment, as well as any outside and cross-project reviewers, in draft form at least one week prior to the due date, so that you have time to incorporate my comments on the drafts into the final versions.  Concurrences shall be obtained where appropriate. 

\subsection{} At the end of the capstone, you will schedule a day to transfer all necessary engineering documentation / knowledge.  This may include: burning code and data to a CD or uploading it to a repository; preparation of system diagrams, wiring diagrams, drawings, etc; providing files used in production (g-code, STL, circuit board layouts, etc.); providing instructions on how to run the demo; turnover of spare materials and supplies and/or equipment return; preparation of a design notebook archiving the entire design history. Appropriate use of the Google Team Drive throughout the proejct will simplify this task considerably.

\section{Management tools} 

\subsection{Work tracking} Establish a method of tracking assigned tasks. This can be hardcopy (e.g. NR greens system, kanban, physical stickies on a board), a spreadsheet, or use of software such as Trello, Asana, Slack. I will not count the hours you work, but I will check that tasks are getting done in a timely manner. 

\subsection{Software tools} You are required to use the Google Team Drive. Github and Latex are encouraged, to reduce rework needed to publish your findings later. 

\subsection{Team spaces} You may establish a physical team space in Maury 224. In addition, electronic spaces have been prepared for you in the Google Team Drive and in Slack. 

\subsection{Agile Project Management} I will encourage the team to try out Agile Project Management methods (e.g. Scrum). To apply the Scrum technique, we will identify items to focus on for 1-2 week sprints; then have focused check-ins every 1-2 days to ask what tasks were done; what tasks are being done today; and what blocks each team member is facing. 

\subsection{Meetings}  Meetings can be very useful, but are also very costly in terms of person-hours.  Make the world a better place, hold good meetings.  Do not waste your meetings! Use them to create buy-in/consensus and to drive towards decisions and actions. Plan to cover status of work, schedule, cost and hours; biggest current challenge and/or technical question you want to discuss; for any design decisions / course of actions you want to resolve at the meeting, provide your recommendations along with supporting information. Written meeting minutes shall be kept documenting who was present, what was discussed, and any decisions or commitments made.  You are also highly encouraged to maintain a design notebook. 

\clearpage
\section{Grading} In a real engineering project, you will be judged based on on-time delivery of something that meets the requirements, at or under budget, along with complete documentation. For this class, CAPT Hurni will decide your grade, but I will provide input based on the following general guidelines:
\begin{center}
\begin{tabular}{lp{5in}}
A & On time, at or under budget, and meeting performance specs \\
B & Behind, but with a path to completion identified \\
C & Critically behind \\
D & Deficient in performance, execution of tasks, planning, or application of engineering principles \\
F & Let's not go here. \\
\end{tabular}
\end{center}

\section{Request for action} The Biomechanics Capstone 2019 team is requested to commence concept studies for tools for biological study enabled by autonomy, machine vision, robotics and control, and drones, subject to the comments above. 

\subsection{} As soon as possible, provide, in writing, a notional division of responsibilities for each engineer on your team (this can changed as needed). 

\subsection{} All team members should have access to the physical space in Maury 224 as well as any electronic spaces / management tools you plan to use. 

\subsection{} Based on recent discussions, immediate priorities are to get in the air as soon as possible; to try out something with machine vision; and to study particular missions and find ways to get to the field to do them. 

\section{}%Cost Paragraph 
The action of this letter is within the scope of ES401. 

\noclosing{}\\
\signspace{}
%\noindent\hspace*{4in}\includesignature{}
\signature{D Evangelista}
\sendertitle{Assistant Professor}

\noindent\hspace*{4in}{235 Maury Hall}\\
\hspace*{4in}{(410) 293-6132}\\
\hspace*{4in}{\emph{evangeli@usna.edu}}


\copyto{}
File (1531)\\
ES401 Instructor, CAPT Hurni\\
Asst Prof Jaramillo\\
MIDN 1/c Descour\\
MIDN 1/c Toth\\
SUAS and SWAT-C faculty\\

%\nobibliography{\myroot/references/course}

\navyrecordnote

\navyrecordnotedistribution{%
Evangelista(w)\\
Hurni(w)\\
Dawkins(w)\\
Devries(w)\\
Kutzer(w)\\
Jaramillo(w)%
}
\navyrecordnoteconcurrences{%
\navyrecordnoteconcurrence{ES401 Instructor}}

\navyrecordnotesubjline

\section{}
This letter provides tasking for the Biomechanics Capstone Team 2019 (Edwards, Hall, Martin, and Nemani). 

\section{}
The team has indicated they are most interested in developing a drone to accomplish a science mission. As of this writing, the missions could include machine vision; robotics for sample return; autonomy; thus the team will need to simplify and downselect what parts require new design engineering versus what can be accomplished with commercial off-the-shelf (COTS) solutions. 

\section{} Nemani is registered for ES495 and intends to present at a conference / in order to connect with scientists to deploy the device in the field. Martin is registered for ES281C to become a drone operator; Nemani already has experience from the DARPA Service Academy Swarm Challenge; and Edwards is a fixed wing R/C pilot. In addition, Edwards and Martin both have some experience in the pre-med track at USNA and thus have some biology background.  
\end{document}


